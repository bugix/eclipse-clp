% BEGIN LICENSE BLOCK
% Version: CMPL 1.1
%
% The contents of this file are subject to the Cisco-style Mozilla Public
% License Version 1.1 (the "License"); you may not use this file except
% in compliance with the License.  You may obtain a copy of the License
% at www.eclipse-clp.org/license.
% 
% Software distributed under the License is distributed on an "AS IS"
% basis, WITHOUT WARRANTY OF ANY KIND, either express or implied.  See
% the License for the specific language governing rights and limitations
% under the License. 
% 
% The Original Code is  The ECLiPSe Constraint Logic Programming System. 
% The Initial Developer of the Original Code is  Cisco Systems, Inc. 
% Portions created by the Initial Developer are
% Copyright (C) 2006 Cisco Systems, Inc.  All Rights Reserved.
% 
% Contributor(s): 
% 
% END LICENSE BLOCK


%----------------------------------------------------------------------
\section{Overview}

\index{abstract machine}
\index{Warren Abstract Machine}
\index{WAM}
\eclipse's abstract machine is a variant of the Warren Abstract
Machine \cite{warren83}. Familiarity with its concepts will help
understanding this section.

The main differences between the original WAM and our machine's
corresponding features are
\begin{itemize}
\item Separate machine words used for tag and value.
\item Separate choicepoint ("control") and environment ("local") stacks.
\item No CP (continuation pointer) register. The return address
        is passed via the local stack.
\item Allocation of temporaries on the local stack rather than in
        (argument) registers.
\item A different scheme (\cite{compnd}) for the compiled
        unification and construction of compound terms.
\item Separate calling conventions for Prolog predicates and external
        predicates (written in the implementation language C).
\item A weaker form of environment trimming.
\end{itemize}
The main additional functionality of the {\eclipse} abstract machine
consists in
\begin{itemize}
\item support for attributed variables
\item support for goal suspension and resuming
\item support for cut and block/exit_block.
\item support for dynamic and parallel choicepoints.
\item support for synchronous event handling, including
        triggering of garbage collection.
\item hooks for a box-model tracer.
\end{itemize}
\index{sepia}
Note that, for historical reasons, the name {\bf sepia} is sometimes
used when talking about the {\eclipse} kernel.


%----------------------------------------------------------------------
\section{Storage Model/Memory Organisation}

\subsection{Stacks}
{\eclipse} stores information in the following memory areas, see figure
\ref{figstacks}:
\index{shared heap}
\index{stack}
\begin{description}
\item[abstract machine descriptor] the abstract machine registers\index{registers}
        (argument registers, stack pointers, etc)
\item[shared heap] abstract machine code, shared ground terms,
        heap copied terms (setval, record, etc)
\item[local stack] return addresses, environment frames
\item[control stack] choice points (copies of parts of the
        abstract machine state)
\item[trail stack] undo information (pointers into the stacks, possibly
        with associated data)
\item[global stack] most variable-sized data (lists, structures,
        strings, bignums, suspended goal descriptors, etc)
\end{description}
\begin{figure}
\epsfbox{stacks.eps}
\label{figstacks}
\caption{{\eclipse} memory areas}
\end{figure}
The following choices have been made in the current implementation:
\begin{itemize}
\item Each stack occupies a consecutive memory area.
\item Stacks are paired (control-local and global-trail) and each pair
        grows towards each other.
        Therefore there is only a common size limit for each pair,
        not for each individual stack.
\item A maximum amount of virtual address space is reserved for each
        stack pair, physical memory is mapped (in reasonably large increments)
        into the virtual space as the stacks grow, and unmapped as they
        shrink.
\end{itemize}
To simplify address comparisons, the abstract machine requires that
\index{local stack}
\index{global stack}
\begin{itemize}
\item the local stack is at higher addresses than the global stack
\item the local stack grows from high to low addresses
\item the global stack grows from low to high addresses
\end{itemize}



\subsection{Data Representation}
\label{chapdatarep}

{\eclipse} uses a tagged\index{tagged}  data representation.
The basic data unit in {\eclipse} is the {\bf pword}\index{pword}.
It consists of a value and tag field, both are the size of a machine
{\bf word} (32 bit or 64 bit, by which we mean the size of
a pointer on that machine). In more detail, the layout of a pword is:
\begin{verbatim}
32-bit:   31   30   29   28           7 6 5 4 3 2 1 0
64-bit:   63   62   61   60           7 6 5 4 3 2 1 0
        +---+----+----+----+---------+---------------+
        |REF|MARK|LINK|MISC|   ...   |      TXXX     |
        +---+----+----+----+---------+---------------+
        |              pointer-sized value           |
        +--------------------------------------------+

        typedef struct s_pword
        {
            value val;                      /* value part first */
            type tag;                       /* then tag part */
        } pword;
\end{verbatim}
The conceptual tag is made up of the REF-bit and the 8-bit-tag TXXX
(where TXXX is TINT for integers, TFLOAT for floats etc).
The MARK\index{MARK}  and LINK\index{LINK}  bits are used by the garbage collector (the MARK
bits also temporarily in routines like term copying).
The MISC bit is used for different purposes with different kinds of tags,
see below.
The remainder of the space in the tag word is reserved, and currently
only used for some variables (to store the variable name),
and dictionary string buffers (to store a reference count).

Other frequently used data type on the implementation level are:
\begin{verbatim}
word             pointer-sized signed integer
uword            pointer-sized unsigned integer
value            a word-sized union of types for values
type             a word-sized union of integers for tags
int32            32-bit signed integer
uint32           32-bit unsigned integer
dident           dictionary identifier (pointer to descriptor)
pri*             procedure identifier (pointer to descriptor)
\end{verbatim}


\subsubsection{Atomic Types, Boxed and Unboxed}
Figure \ref{figatomic} gives an overview of the atomic data types of
{\eclipse}.
\begin{figure}
\epsfbox{ecatomic.eps}
\label{figatomic}
\caption{{\eclipse} atomic data types}
\end{figure}
\index{boxed}
Constants whose value does not fit into a word are {\it boxed}, i.e. the
value part of the pword contains a pointer to a buffer, which in turn
holds the constant's value. The buffer\index{buffer}  is usually on the global stack,
but may also be in the heap (e.g. for constants\index{constants}  occurring in program code).
Stack buffers have a pword-sized
header, consisting of a TBUFFER\index{TBUFFER}  tag and a value indicating the number of
valid bytes in the buffer (minus 1).  The physical size of the buffer is
this content size rounded up to a multiple of a pword.

\subsubsection{Small Integers}
Signed two's complement integers\index{integers}  up to the machine's wordsize are an
atomic types and are represented with a TINT\index{TINT}  tag.  Larger integers are
represented as {\em bignums}.

\subsubsection{Atoms}
Atoms\index{Atoms}  are represented with a TDICT\index{TDICT}  tag, the value part being a 
pointer to their dictionary entry (dident, see \ref{chapdictionary}).
An exception is the nil atom '[]', which has its own tag TNIL\index{TNIL}  and
an undefined value part. The reason to have the TNIL tag is to speed
up list operations by having only to deal with TLIST and TNIL tags.

\subsubsection{Floats/Doubles}
Older {\eclipse} versions supported both single and double precisions floats.
This is no longer the case, the single float type has been dropped.
On 64-bit machines, doubles\index{doubles}  are represented like small integers, with
a TDBL tag, and the value part consisting of the actual double value.
On 32-bit machines, doubles are {\it boxed}, i.e. the value part contains
a pointer to a global stack buffer which then holds the actual double value.
A boxed double therefore occupies 3 pwords: the TDBL\index{TDBL}-tagged pointer, the
TBUFFER-tagged buffer header, and a pword-sized buffer holding the double
value itself.
The implementation uses the UNBOXED_DOUBLES\index{UNBOXED_DOUBLES}  macro to distinguish between
the two possible representations.
In the boxed case, the TDBL tag may have the PERSISTENT\index{PERSISTENT}  bit set (see
ground constant optimisation \ref{secgroundconst}).

\subsubsection{Bounded Reals}
Bounded reals (breals\index{breals}) are an {\eclipse} specific data type consisting
of two doubles representing an interval.  The are stored like
boxed doubles, except that the buffer contains two doubles.
The tag is TIVL\index{TIVL}.  Normally, the breal is canonical, i.e.\ the lower bound
is not larger that the upper bound.  If this is not the case, the
RAW_IVL\index{RAW_IVL}  (=MISC) bit  is set in the buffer tag (the lexical analyser can
produce such objects).
The TIVL tag may have the PERSISTENT bit set (see ground constant optimisation
\ref{secgroundconst}).

\subsubsection{Strings}
Unlike many Prolog system, {\eclipse} has true strings.
Strings\index{Strings}  are always {\it boxed}. The tag is TSTRG\index{TSTRG}, and the value is a pointer
to a global stack buffer holding the actual string.
Even though the buffer header contains an explicit length field,
strings are additionally zero-terminated in order to be
downward-compatible with C strings\index{strings}.  As long as the strings are only
manipulated using {\eclipse}'s own string primitives, strings may
contain embedded NUL\index{NUL}  bytes.  {\eclipse} strings are conceptually
sequences of bytes, not characters in a particular encoding\index{encoding}.
The TSTRG tag may have the PERSISTENT bit set (see ground constant optimisation
\ref{secgroundconst}).
The string buffer may have the IN_DICT\index{IN_DICT}  (=MISC) bit set, meaning that the
string is part of the dictionary (see \ref{chapdictionary}).


\subsubsection{Bignums}
The main pword for a bignum has a TBIG\index{TBIG}  tag and a value pointing to a
standard global stack buffer. 
Bignum\index{Bignum}  (and rational) computations are delegated to the Gmp\index{Gmp}  (Gnu
multi-precision, www.swox.com/gmp) library.  Gmp's limb array is stored
in the global stack buffer. The number's sign is stored as the BIGSIGN\index{BIGSIGN}  (=MISC)
bit in the buffer header tag. The library's MP_INT\index{MP_INT}  or MP_RAT\index{MP_RAT}  structure only
gets created temporarily in order to pass the number to a gmp
function.  Normally, computation results are always normalised such that
word-sized integers are stored as small (TINT) integers, and bignums are
always too large to fit into a word.  This rule is only violated
temporarily (the bignum/2\index{bignum/2} predicate can create a short bignum in order
to convert a TINT x TBIG operation into a TBIG x TBIG one, see arithmetic
type coercion).
The TBIG tag may have the PERSISTENT bit set (see ground constant optimisation
\ref{secgroundconst}).

\subsubsection{Rationals}
Rationals\index{Rationals}  have a TRAT\index{TRAT}  tag and a consecutive pair of TBIG pwords on the
global stack, representing the numerator\index{numerator}  and denominator\index{denominator}  (these
bignums can actually be small integers, since it is probably not worth
optimising this case).  Rational computations are delegated to the Gmp
library, whose MP_RAT structure only gets created temporarily in order
to pass the number to a gmp function.
The TRAT tag may have the PERSISTENT bit set (see ground constant optimisation
\ref{secgroundconst}).

\subsubsection{External Data Handles}
The handle type is intended to store references to non-{\eclipse} data.
It consists of a THANDLE-tagged pointer to a pair of pwords on the global
stack. The first on has a TEXTERN tag and a pointer to a type descriptor
(struct t_ext_type) as specified in the Embedding Manual. The second one
has a TPTR tag and pointer to arbitrary user-defined data.

\subsubsection{Compound types (Lists and Structures)}
Figure \ref{figcompound} shows the compound data types.
\begin{figure}
\epsfbox{eccomp.eps}
\label{figcompound}
\caption{{\eclipse} compound data types}
\end{figure}
Lists\index{Lists}  are represented by a TLIST\index{TLIST}-tagged pointer, pointing to a consecutive
pair of pwords on the global stack, representing the list head\index{head}  and tail\index{tail}.

General structures\index{structures}  are represented by a TCOMP\index{TCOMP}-tagged pointer pointing to
consecutive pwords on the global stack, of which the first one represents
the functor\index{functor}, and the following ones are the arguments from 1 to n (arity).
The functor pword is similar to an atom, it has a TDICT\index{TDICT}  tag and a dident
value.  The arity\index{arity}  can be looked up from the dident dictionary entry.
While the arity for atoms is always 0, the arity for compound terms is
always greater than 0.  There is no artificial upper limit for the arity.

Both TLIST and TCOMP tags may have the PERSISTENT\index{PERSISTENT}  bit set when they point
to ground data structures (see ground constant optimisation \ref{secgroundconst}).

\subsubsection{Variables and References}
\begin{figure}
\epsfbox{ecvars.eps}
\caption{{\eclipse} variable and reference types}
\end{figure}
References are distinguished from free variables\index{variables}  only by their value part:
A self reference\index{self reference}  is a free variable\index{variable}, otherwise a reference\index{reference}  (an indirection)
pointing to another word.
All variable tags have the TREFBIT\index{TREFBIT}  (REF in the picture) set\footnote{
This holds as of version 5.2. Previously, only the simple variable had
the TREFBIT set}.

In the reference (non-self-reference) case the rest of the tag is
irrelevant.  In the self-reference case, the rest of the tag indicates
the kind of variable (simple, named\index{named variable}, attributed, etc).
Note that this scheme has the advantage that pwords can be copied
regardless of their content: when copied, a variable automatically
turns into a reference to the original variable (rather than creating
a new, different variable).

Variables with TNAME\index{TNAME}  and TMETA\index{TMETA}  tags reserve a 19-bit field in the tag
for storing a variable name\index{variable name}  (mainly for debugging purposes).


\subsubsection{Attributed Variables}

Attributed variables\index{attributed variables}  consist of a TMETA\index{TMETA}-tagged self-reference pword,
followed by a TCOMP-tagged pword representing a compound term with
functor meta/N, which holds the N attributes.

The HIDE_ATTR\index{HIDE_ATTR}  (=MISC) bit in TMETA tags is used as a marking bit to
avoid looping during printing of attributes.



\subsection{Creating Data}

Data is created either by executing suitable abstract machine
instructions, or by external (C, C++) code using the external
interface macros/functions.

\index{put instruction}
\index{get instruction}
The data-creating abstract machine instructions are essentially the
{\bf Put} family, the {\bf Get} family in write mode and the
{\bf Out_get} family.
\begin{description}
\item[Put/Get_constant] creates any atomic type in one of the
        machine's argument registers.
        There are a number of specialised instructions
        (Get/Put_integer/atom/nil/...) for the most common data types.
        In case of the types that don't
        fit into a single word (strings, bignums), the data buffer is
        located on the shared heap. The instruction just loads a pointer
        to this shared buffer into the argument register, rather than
        copying the buffer onto the global stack.
\item[Put/Get_list/structure]
        Structures and lists get created according to
        \cite{compnd}. The head unification uses Get_list/structure
        instructions followed by separate read and write sequences
        ({\bf Read} and {\bf Write} instructions), the body
        argument construction uses Put_list/structure instructions
        followed by instructions of the {\bf Push} family.
\end{description}


%----------------------------------------------------------------------
\section{Abstract Machine Registers}
%----------------------------------------------------------------------

The `registers' of the abstract machine\index{abstract machine}  are fields in the global
data structure {\tt ec_.m} of type {\tt struct machine}.
While the emulator\index{emulator}  is running, some of these conceptual registers
are cached in local variables of the emulator function ec_emulate().

\subsection{Basic Stack Management}
\index{SP}
\index{B}
\index{TT}
\index{TG}
\index{SP_LIMIT}
\index{B_LIMIT}
\index{TT_LIMIT}
\index{TG_LIMIT}
\begin{sloppypar}
\begin{description}
\item[SP] top of local stack. The stacks grows towards lower addresses,
        SP points to the top word. Word-aligned, contains a mixture
        of pwords, saved E and saved PP registers.
\item[B] top of control stack. The stack grows towards higher addresses.
        B points to the first free word. Word-aligned, contains a mixture
        of pwords and saved engine registers.
\item[TT] top of trail stack. The stacks grows towards lower addresses,
        TT points to the top word. Word-aligned, contains a mixture
        of pwords and addresses and other words.
\item[TG] top of global stack. The stack grows towards higher addresses.
        TG points to the first free pword, the stack is pword-aligned.
        Contains only items listed in \ref{chapdatarep}.
\item[SP_LIMIT] allocation limit for the local stack. When SP crosses this
        boundary, the local stack needs to be expanded immediately.
        There is only a small margin of LOCAL_CONTROL_GAP between SP_LIMIT
        and the end of the mapped memory.
\item[B_LIMIT] allocation limit for the local stack. When B crosses this
        boundary, the control stack needs to be expanded immediately.
        There is only a small margin of LOCAL_CONTROL_GAP between B_LIMIT
        and the end of the mapped memory.
\item[TT_LIMIT] allocation limit for the trail stack. When TT crosses this
        boundary, the trail stack needs to be expanded immediately.
        There is only a small margin of TRAIL_GAP between TT_LIMIT
        and the end of the mapped memory.
\item[TG_LIMIT]
	allocation limit for the global stack. When TG crosses this
        boundary, the global stack needs to be expanded immediately.
        There is only a small margin of GLOBAL_TRAIL_GAP between TG_LIMIT
        and the end of the mapped memory.
\end{description}
\end{sloppypar}


\subsection{Deterministic execution}
\index{PP}
\index{E}
\index{A1..An}
\index{S}
\index{EXPORTED}
\begin{description}
\item[PP] program (code) pointer, points to next abstract machine instruction.
\item[A1..A256] argument registers. Their number limits the maximum
        arity of a predicate in \eclipse (but not the arity of
        compound terms!).
\item[E] current environment, points into the local stack.
\item[S] structure pointer, used during unification and creation of
        compound terms.
\item[EXPORTED flag] abstract machine registers are exported from emulator,
        i.e.\ they are not cached in emulator registers.
\end{description}


\subsection{Nondeterministic execution}
\index{B}
\index{TT}
\index{EB}
\index{GB}
\index{LCA}
\index{DET}
\begin{description}
\item[B] top of control stack. The stack grows towards higher addresses.
        B points to the first free word. Word-aligned, contains a mixture
        of pwords and saved engine registers.
\item[TT] top of trail stack. The stacks grows towards lower addresses,
        TT points to the top word. Word-aligned, contains a mixture
        of pwords and addresses and other words.
\item[EB] environment backtrack pointer, points into the local stack.
        caches the value of E in the topmost choice point.
\item[GB] global stack backtrack pointer, points into the global stack.
        caches the value of TG in the topmost choice point.
\item[LCA] last cut action. Top of a conceptual stack of {\it cut action}
        descriptors, implemented as a list threaded into the global stack.
\item[DET flag] no-choicepoint-flag: flag that indicates that
        no choicepoint was created since procedure entry. It becomes invalid
        after the first subgoal call.
\end{description}

\subsection{Suspend/Resume mechanism}
In the following, `volatile' means that the register contents is short-lived
and never gets saved and restored.
\index{LD}
\index{MU}
\index{DE}
\index{SV}
\index{WL}
\index{WP}
\begin{description}
\item[LD] top of the list of all suspended goals. This is a conceptual stack,
        threaded into the global stack (i.e.\ it is a linked list of frames
        with the links going strictly from newer to older frames in the stack).
\item[MU] list of meta-unifications. A volatile register that passes a list
        of attribute-value-pairs from the unification routine to the subsequent
        meta-unify event handler.
\item[DE] current suspension, volatile register to pass the address of
        a suspension that needs re-suspending or waking.
\item[SV] list of suspending variables. A volatile register that passes
        a list from a C-builtin to the emulator code that created a
        suspension for the builtin.
\item[WL] points to the array (1..SUSP_MAX_PRIO) of woken lists
        (a structure on the global stack).
\item[WP,WP_STAMP] current execution priority. Only suspensions with higher
        priority can interrupt the execution. WP is a tagged 
        pword, and is paired with the WP_STAMP because it gets
        value-trailed on change.
\end{description}


\subsection{Events and garbage collection}
\index{GCTG}
\index{TG_SL}
\index{TG_SLS}
\index{IFOFLAG}
\index{GLOBVAR}
\index{ALLREFS}
\index{GLOBAL_NO_IT}
\index{NO_EXIT}
\index{WAS_EXIT}
\index{EVENT_POSTED}
\index{DEL_IRQ_POSTED}
\begin{description}
\item[GCTG] indicates the global stack address corresponding to the
        topmost choicecpoint which already existed
        during the last garbage collection. Everything older than this
        does not need to be garbage collected again.
\item[TG_SL] (TG soft limit) garbage collection and general event trigger.
        This register normally points above the global stack top
        (but within the already memory-mapped area). When the global stack
        top TG crosses this boundary, a garbage collection is triggered.
        The mechanism is also abused to trigger all other synchronous
        engine events (by forcing TG_SL to zero).
        This has the advantage that the execution overhead
        for event handling is restricted to a simple $TG >= TG_SL$ test.
        Unless when it is zero, TG_SL is always between TG and TG_LIMIT.
\item[TG_SLS] (TG_SL shadow) when TG_SL is nonzero, TG_SLS has the same value.
        When TG_SL is zero (a fake overflow, i.e.\ a general synchronous event),
        TG_SLS keeps its original value, which is then used to reset TG_SL
        after event handling.
\item[IFOFLAG] a synchronisation flag used for mutual exclusion when
        TG_SL is changed asynchronously from within an interrupt (signal)
        handler.
\item[GLOBVAR] points to the array (1..GLOBAL_VARS_NO) of \eclipse\
        "global references" (a structure on the global stack).
\item[ALLREFS] points to a list of eclipse_ref_ data structures, i.e.\
        objects holding additional potential references to {\eclipse} data
        from within code written using the embedding interface
        (see ec_refs in the Embedding Manual).
\item[GLOBAL_NO_IT flag] means that interrupts are currently disabled.
\item[NO_EXIT flag] means that preemption via exit_block/1 is currently
        prohibited (e.g.\ because the gc is running).
\item[WAS_EXIT flag] an exit_block was attempted while NO_EXIT was set
\item[EVENT_POSTED flag] the synchronous event-queue contains events
\item[DEL_IRQ_POSTED flag] there is possibly a delayed interrupt among
        posted events
\end{description}


\subsection{Parallelism}
A number of additional registers can be found in the code.
They are mainly related to parallelism\index{parallelism}, which is currently unsupported.


%----------------------------------------------------------------------
\section{Instruction Set}
%----------------------------------------------------------------------

The arguments of the instructions\index{instructions}  are denoted as follows:

\begin{tabular}{|l|l|l|}
\hline
a(A)            & address & argument register A (1..255)\\
y(Y)            & offset & environment slot Y (offset from E register)\\
t(X)            & offset & temporary X (offset from SP register)\\
ref(L)          & address & reference to code address L\\
N               & integer & environment size\\
I               & integer & number, e.g. arity\\
F               & dident & functor (dictionary identifier)\\
C               & pword & simple tagged Prolog word\\
V               & value & value part of Prolog word only\\
P               & proc & procedure identifier\\
D               & bool & debug flag\\
T               & tag & tag (possibly including variable name) \\
\hline
\end{tabular}

\subsubsection{Simple Moves}
These move\index{move}  one pword from a source location to a destination.

%\begin{tabular}{|l|l|}
\begin{tabular}{|p{0.35\textwidth}|p{.6\textwidth}|}
\hline
move(a(A))              & push a(A) onto local stack \\
move(a(A1),a(A2))       & move from argument to argument  \\
move(a(A),y(Y))         & move from argument to environment       \\
move(y(Y),a(A))         & move from environment to argument      \\
move(t(X),a(A))         & move from temporary to argument      \\
move(y(Y),y(Y))         & move from environment to environment      \\
\hline
get_variable(N,a(A),y(Y))& allocate(N) + move(a(A),y(Y))        \\
\hline
\end{tabular}

\subsubsection{General Unification}
These access two general pwords and invoke the general unification\index{unification}  routine.
As a result, failure can occur.  In case of success, the MU\index{MU}  register may
hold a list of attributed variable that were unified. If so, this will
trigger a meta-unify\index{meta-unify}  event at the next synchronous point
(see \ref{seceventhandling}).
\index{get instruction}

%\begin{tabular}{|l|l|}
\begin{tabular}{|p{0.35\textwidth}|p{.6\textwidth}|}
\hline
get_value(a(A1),a(A2))          & unify two argument registers             \\
get_value(a(A),y(Y))            & unify argument and environment slot                \\
get_value(a(A),t(X))            & unify argument and temporary                \\
get_value(y(Y),y(Y))            & unify two environment slots               \\
\hline
\end{tabular}

\subsubsection{Simple Unification}
Unify argument register content with a constant:
\index{get instruction}

%\begin{tabular}{|l|l|}
\begin{tabular}{|p{0.35\textwidth}|p{.6\textwidth}|}
\hline
get_constant(a(A),C)    & unify argument with arbitrary constant \\
get_nil(a(A))           & unify argument with nil (the atom '[]') \\
get_integer(a(A),V)     & unify argument with a (short) integer constant \\
get_float(a(A),V)       & unify argument with a float constant \\
get_atom(a(A),V)        & unify argument with an atom \\
get_string(a(A),V)      & unify argument with a string constant \\
\hline
in_get_constant(a(A),C) & special case for input mode (a(A) instantiated) \\
in_get_nil(a(A))        & special case for input mode (a(A) instantiated) \\
in_get_integer(a(A),V)  & special case for input mode (a(A) instantiated) \\
in_get_float(a(A),V)    & special case for input mode (a(A) instantiated) \\
in_get_atom(a(A),V)     & special case for input mode (a(A) instantiated) \\
in_get_string(a(A),V)   & special case for input mode (a(A) instantiated) \\
\hline
out_get_constant(a(A),C)& special case for output mode (a(A) uninstantiated) \\
out_get_nil(a(A))       & special case for output mode (a(A) uninstantiated) \\
out_get_integer(a(A),V) & special case for output mode (a(A) uninstantiated) \\
out_get_float(a(A),V)   & special case for output mode (a(A) uninstantiated) \\
out_get_atom(a(A),V)    & special case for output mode (a(A) uninstantiated) \\
out_get_string(a(A),V)  & special case for output mode (a(A) uninstantiated) \\
\hline
\end{tabular}

\subsubsection{Structure Unification}
Compilation of compound\index{compound}  term unification is described in some more
detail in section \ref{secstructunify}. The compiler generated a code sequence
for unification of a possibly nested structure. The instructions starting
these sequences are:
\index{get instruction}

%\begin{tabular}{|l|p{0.6\textwidth}|}
\begin{tabular}{|p{0.35\textwidth}|p{.6\textwidth}|}
\hline
get_structure(a(A),F,ref(L))    & If argument is variable, push new structure
        frame, set S register, and continue. If nonvariable, check if
        structure with functor F, set S and jump to label L. Otherwise fail \\
\hline
get_list(a(A),ref(L))           & special case for list   \\
in_get_structure(a(A),F,ref(L))         &        special case for input mode      \\
in_get_list(a(A),ref(L))        & special case for list in input mode             \\
out_get_structure(a(A),F)       & special case for output mode                    \\
out_get_list(a(A))              &  special case for list in output mode           \\
get_structure_arguments(a(A))   & special case input mode after switch            \\
get_list_arguments(a(A))        & special case input mode after switch            \\
\hline
\end{tabular}

If the input argument is instantiated to a structure, the structure
arguments are matched or deconstructed with read instructions:
\index{read instruction}

%\begin{tabular}{|l|l|}
\begin{tabular}{|p{0.35\textwidth}|p{.6\textwidth}|}
\hline
read_constant(C)          & match constant with location S++ \\
read_nil                  & match nil constant with location S++ \\
read_integer(C)           & match integer constant with location S++ \\
read_float(C)             & match float constant with location S++ \\
read_atom(C)              & match atom constant with location S++ \\
read_string(C)            & match string constant with location S++ \\
read_void                 & increment S \\
read_void(N)              & increment S by N \\
read_variable(a(A))       & move content of location S++ to argument \\
read_variable(y(Y))       & move content of location S++ to environment \\
read_variable(N,y(Y))     & move content of location S++ to new environment \\
read_variable             & move content of location S++ to new temporary  \\
read_value(a(A))          & unify argument with location S++ \\
read_value(y(Y))          & unify environment slot with location S++ \\
read_value(t(X))          & unify temporary with location S++ \\
read_reference(a(A))      & create reference to S++ in argument \\
read_reference(y(Y))      & create reference to S++ in environment \\
read_reference(N,y(Y))    & create reference to S++ in new environment \\
read_reference            & create reference to S++ in new temporary \\
\hline
\end{tabular}

If the input argument is uninstantiated, the structure arguments get
constructed by a sequence of write instructions. It is assumed that the
preceding get/write/read_structure/list instruction has constructed the
structure frame, and set the S register pointing to the arguments that
need to be filled in by the write sequence:
\index{write instruction}

%\begin{tabular}{|l|l|}
\begin{tabular}{|p{0.35\textwidth}|p{.6\textwidth}|}
\hline
write_constant(C)          & write constant to S++ \\
write_nil                  & write nil constant to S++ \\
write_integer(C)           & write integer constant to S++ \\
write_float(C)             & write float constant to S++ \\
write_atom(C)              & write atom constant to S++ \\
write_string(C)            & write string constant to S++ \\
write_void                 & create free variable at S++ \\
write_variable             & create free variable at S++, put reference in new temporary  \\
write_variable(a(A))       & create variable at S++, put reference in argument \\
write_variable(y(Y))       & create variable at S++, put reference in environment \\
write_variable(N,y(Y))     & create variable at S++, put reference in new environment \\
write_value(a(A))          & move argument to location S++ \\
write_value(y(Y))          & move environment slot to location S++ \\
write_value(t(X))          & move argument to location S++ \\
write_named_void(T)             & create named free variable at S++ \\
write_named_variable(T)         & create named free variable at S++, put reference in new temporary  \\
write_named_variable(a(A),T)    & create named variable at S++, put reference in argument \\
write_named_variable(y(Y),T)    & create named variable at S++, put reference in environment \\
write_named_variable(N,y(Y),T)  & create named variable at S++, put reference in new environment \\
write_local_value(a(A))         & move argument to location S++, globalising if necessary \\
write_local_value(y(Y))         & move environment slot to location S++, globalising if necessary \\
write_local_value(t(X))         & move temporary to location S++, globalising if necessary \\
\hline
\end{tabular}

Since nested structures are handled depth-first, a mechanism is required
to keep track of the nesting level. Moreover, it is possible that a subterm
of a read-mode structure is in write mode, so the mode may need to be changed
when switching level up or down. Both these information bits are held in
a per level temporary register.
The read mode instructions are:

%\begin{tabular}{|l|l|}
\begin{tabular}{|p{0.35\textwidth}|p{.6\textwidth}|}
\hline
read_structure(F,ref(L))        & unify first compound subterm at a nesting level \\
read_structure(F,t(X),ref(L))   & unify compound subterm after simple subterm \\
read_next_structure(F,t(X),ref(L)) & unify compound subterm after compound subterm \\
read_last_structure(F,ref(L))   & unify compound subterm which is last in term \\
\hline
read_list(ref(L))               & unify first list subterm at a nesting level \\
read_list(t(X),ref(L))          & unify list subterm after simple subterm \\
read_next_list(x(X),ref(L))     & unify list subterm after compound subterm \\
read_last_list(ref(L))          & unify list subterm which is last in term \\
%\hline
%read_meta(T,ref(L))             & unify first attr variable at a nesting level \\
%read_next_meta(t(X),T,ref(L))   & unify attr variable after simple subterm \\
%read_meta(t(X),T,ref(L))        & unify attr variable after compound subterm \\
%read_last_meta(T,ref(L))        & unify attr variable which is last in term \\
\hline
mode(t(X))                      & continue at higher nesting level \\
\hline
\end{tabular}

And the write mode instructions are:

%\begin{tabular}{|l|l|}
\begin{tabular}{|p{0.35\textwidth}|p{.6\textwidth}|}
\hline
write_structure(F)              & write reference to new structure frame, and continue there \\
write_list                      & write reference to new list cell, and continue there \\
write_meta(T)                   & write reference to attr variable, and continue there \\
\hline
first                           & prefix for write first subterm at a nesting level \\
next(t(X))                      & prefix for write subterm after compound subterm \\
next(t(X),ref(L))               & prefix for write subterm after compound subterm \\
mode(t(X),ref(L))               & continue at higher nesting level, maybe read mode \\
\hline
\end{tabular}


\subsubsection{Head Matching}
These are used (together with the in_get family and the read
instructions) to compile matching clauses, i.e.\ one-way unification
that do not instantiate anything in the caller arguments.
\index{matching instruction}

%\begin{tabular}{|l|l|}
\begin{tabular}{|p{0.35\textwidth}|p{.6\textwidth}|}
\hline
get_matched_value(a(A1),a(A2))  & match two arguments \\
get_matched_value(a(A),y(Y))    & match argument and environment slot \\
get_matched_value(a(A),t(X))    & match argument and temporary \\
read_test_var                   & fail if location S holds a variable \\
\hline
\end{tabular}

Matching clauses are the only way to match and access the attributes of
attributed variables. Special instructions are used to implement this:

%\begin{tabular}{|l|l|}
\begin{tabular}{|p{0.35\textwidth}|p{.6\textwidth}|}
\hline
in_get_meta(a(A),ref(L))        & check for attr variable in argument, set S \\
%read_next_meta(t(X),T,ref(L))   &                         \\
%read_meta(t(X),T,ref(L))        &                         \\
%%read_last_meta(T,ref(L))        &                         \\
%read_matched_value(a(A))        &                         \\
%read_matched_value(y(Y))        &                         \\
%read_matched_value(t(X))        &                         \\
match_meta                      & check for attr variable at nesting level \\
match_next_meta(t(X))           & check for attr variable after simple subterm \\
match_meta(t(X))                & check for attr variable after compound subterm \\
match_last_meta                 & check for attr variable which is last in term \\
%read_meta(T,ref(L))             &                 \\
read_attribute(At)              & expect attribute structure at S, set S to requested attribute \\
\hline
\end{tabular}


\subsubsection{Regular subgoal argument construction}
The arguments for regular predicate calls (calling convention requires
arguments in arguments registers) are loaded using the Put family of
instructions:
\index{put instruction}

%\begin{tabular}{|l|p{10cm}|}
\begin{tabular}{|p{0.35\textwidth}|p{.6\textwidth}|}
\hline
put_variable(a(A),y(Y))         & move y(Y) to a(A)               \\
put_variable(a(A))              & set a(A) to new free variable on global  \\
put_unsafe_value(a(A),t(X))     & move t(X) to a(A), globalise if needed                          \\
put_unsafe_value(a(A),y(Y))     & move y(Y) to a(A), globalise if needed                          \\
put_constant(a(A),C)            & move constant to a(A)           \\
put_nil(a(A))                   & move nil constant to a(A)              \\
put_integer(a(A),C)             & move small integer constant to a(A)              \\
put_float(a(A),C)               & move float constant to a(A)              \\
put_atom(a(A),C)                & move atom constant to a(A)              \\
put_string(a(A),C)              & move string constant to a(A)              \\
put_list(a(A))                  & push list skeleton, pointed to by a(A) and S          \\
put_structure(a(A),F)           & push structure skeleton with functor F,
                                let a(A) point to it, let S point to first argument\\
%\hline
%\multicolumn{2}{|c|}{the next 2 are strange...}\\
\hline
put_reference(a(A),O,T) & push O bytes onto global, init first pword
            with self reference with tag T, refer to it from a(A) and S\\
put_reference(a(A),y(Y),O,T) & push O bytes onto global, init first
            pword with self reference with tag T, refer to it from
            a(A),y(Y) and leave S pointing behind it\\
\hline
\end{tabular}

Arguments of structures constructed with Put-instructions are created
with Push instructions. Many of these are alias of the corresponding
Write instructions, they construct data at the location pointed to by the
S register.
\index{push instruction}

%\begin{tabular}{|l|l|}
\begin{tabular}{|p{0.35\textwidth}|p{.6\textwidth}|}
\hline
push_void                       & same as write_void \\
push_variable(a(A))             & same as write_variable \\
push_variable(y(Y))             & same as write_variable \\
push_init_variable(y(Y))        & create variable at S++, put reference in environment (trailed) \\
push_variable                   & same as write_variable \\
push_self_reference(I)          & create named variable at S++ \\
push_void_reference(O)          & push new nonstandard variable frame, write reference to S++ \\
push_reference(O)               & push new nonstandard variable frame, write reference to S++ and new temporary \\
push_reference(a(A),O)          & push new nonstandard variable frame, write reference to S++ and argument \\
push_reference(y(Y),O)          & push new nonstandard variable frame, write reference to S++ and environment slot \\
push_init_reference(y(Y),O)     & push new nonstandard variable frame, write reference to S++ and environment slot (trailed) \\
push_value(a(A))                & same as write_value \\
push_value(y(Y))                & same as write_value \\
push_value(t(X))                & same as write_value \\
push_value(G)                   & write reference to S+G \\
push_local_value(a(A))          & like write_local_value, but no occur check test \\
push_local_value(y(Y))          & like write_local_value, but no occur check test \\
push_local_value(t(X))          & like write_local_value, but no occur check test \\
push_constant                   & same as write_constant \\
push_nil                        & same as write_nil \\
push_integer(C)                 & same as write_integer \\
push_float(C)                   & same as write_float \\
push_string(C)                  & same as write_string \\
push_list                       & create new list frame, write pointer to S++ \\
push_structure(I)               & create new structure frame, write pointer to S++ \\
\hline
\end{tabular}


\subsubsection{Simple subgoal argument construction}
The Puts family of instructions prepares arguments for calls to predicates
using the external calling conventions, i.e.\ the arguments are dereferenced
and passed over the local stack.
Arguments of constructed structures use the same push instructions as above.
\index{puts instruction}

%\begin{tabular}{|l|l|}
\begin{tabular}{|p{0.35\textwidth}|p{.6\textwidth}|}
\hline
puts_variable           & push free variable on local stack \\
puts_variable(y(Y))     & init environment slot and push reference onto local stack \\
puts_reference(O,T)     & create new nonstandard variable with tag T, push reference onto local stack, set S to it \\
puts_reference(y(Y),O,T)& create new nonstandard variable with tag T, push reference onto local stack and environment slot, set S to it \\
puts_value(a(A))        & push dereferenced content of argument onto local stack \\
puts_value(y(Y))        & push dereferenced content of environment slot onto local stack \\
puts_value(t(X))        & push dereferenced content of temporary onto local stack \\
puts_value(G)           & push reference to S+G onto local stack \\
puts_constant           & push constant onto local stack \\
puts_nil                & push nil constant onto local stack \\
puts_integer(C)         & push small integer constant onto local stack \\
puts_float(C)           & push float constant onto local stack \\
puts_atom(C)            & push atom constant onto local stack \\
puts_string(C)          & push string constant onto local stack \\
puts_list               & create new list frame, push pointer onto local stack, set S \\
puts_structure(D)       & create new list frame, push pointer onto local stack, set S \\
puts_proc(P)            & push a tagged procedure identifier \\
\hline
\end{tabular}



\subsubsection{Choicepoints}
The basic set of choicepoint instructions is for clause choicepoints.
The variants are for the different cases of the current clause, the
alternative, or none of them following inline.
Some more detail is in section \ref{secnondet}.
\index{try instruction}

%\begin{tabular}{|l|l|}
\begin{tabular}{|p{0.35\textwidth}|p{.6\textwidth}|}
\hline
try_me_else(D,I,ref(L))         & create choicepoint, first alternative follows inline, next at L \\
try(D,I,ref(L))                 & create choicepoint, first alternative at L, next follows inline \\
try(D,I,ref(La),ref(L))         & create choicepoint, first alternative at La, next at L \\
retry_me_else(D,ref(L))         & restore choicepoint, this alternative follows inline, next at L \\
retry(D,ref(L))                 & restore choicepoint, this alternative at L, next follows inline \\
retry(D,ref(La),ref(L))         & restore choicepoint, this alternative at La, next at L \\
trust_me(D)                     & restore and pop choicepoint, last alternative follows inline \\
trust(D,ref(L))                 & restore and pop choicepoint, last alternative at L \\
\hline
failure                 & goto failure continuation \\
refail                  & pop one choicepoint and fail again \\
\hline
\end{tabular}

Choicepoints for inlined disjunctions have an environment size parameter
instead of predicate arity.

\begin{tabular}{|p{0.35\textwidth}|p{.6\textwidth}|}
\hline
try_me_inline(D,ref(L),N)       & like try_me_else\\
retry_me_inline(D,ref(L),N)     & like retry_me_else\\
trust_me_inline(D,N)            & like trust_me\\
\hline
\end{tabular}

If the condition is simple, the if-then-else construct is compiled with
small choicepoints, consisting only a the BP and PB field. These are
manipulated using 3 instructions:

\begin{tabular}{|p{0.35\textwidth}|p{.6\textwidth}|}
\hline
set_bp(ref(L))          & save failure continuation (push small choicepoint) \\
restore_bp              & restore failure continuation (pop small choicepoint) \\
new_bp(ref(L))          & update failure continuation (modify small choicepoint) \\
\hline
\end{tabular}

Dynamic predicates have special choicepoint instructions, to allow for
runtime modification and garbage collection of dead clauses.

\begin{tabular}{|p{0.35\textwidth}|p{.6\textwidth}|}
\hline
try_me_dynamic(Tb,Td,ref(L),I)  & create dynamic predicate choicepoint \\
retry_me_dynamic(Tb,Td,ref(L),I)& restore from dynamic predicate choicepoint \\
\hline
\end{tabular}


\subsubsection{Switches}
Compared to a basic WAM, {\eclipse} employs a list_switch instruction
(a simpler form of switch_on_type for the common case of list
processing predicates), and the integer_range_switch which has cases
for lower and upper bounds. All switch instructions refer to an associated
value-label table elsewhere in memory, except switch_on_type and list_switch,
where the small fixed-size table is part of the instruction.
All switch instructions continue inline when the argument is a variable
(except for switch_on_type, which has a special label for attributed
variables).
\index{switch instruction}

%\begin{tabular}{|l|l|}
\begin{tabular}{|p{0.5\textwidth}|p{.45\textwidth}|}
\hline
switch_on_type(a(A),Labels...)           & branch according to tag \\
list_switch(a(A),ref(Ll),ref(Ln),ref(Ld))& branch according to tag \\
integer_switch(a(A),Table,ref(Ld))       & branch according to integer value \\
integer_range_switch(a(A),Table,ref(Le),ref(Ld)) & branch according to integer value \\
atom_switch(a(A),Table,ref(Ld))          & branch according to atom value \\
functor_switch(a(A),Table,ref(Ld))       & branch according to functor value \\
\hline
\end{tabular}

\subsubsection{Call/Return}
Instructions for predicate call and return. These exist in several
variants and combinations. See also section \ref{seccallret}.
\index{call instruction}
\index{ret instruction}
\index{jmp instruction}

%\begin{tabular}{|l|p{10cm}|}
\begin{tabular}{|p{0.35\textwidth}|p{.6\textwidth}|}
\hline
call(ref(L),N)          & predicate call via label \\
call(P,N)               & predicate call via procedure desc \\
callf(ref(L),N)         & predicate call via label (first subgoal), sets DET flag \\
callf(P,N)              & predicate call via procedure id (first subgoal), sets DET flag \\
jmp(ref(L))             & predicate tail call via label \\
jmp(P)                  & predicate tail call via procedure id \\
jmpd(ref(L))            & predicate tail call (assume DET still set) \\
jmpd(P)                 & predicate tail call via procedure id (assume DET still set) \\
jmpd(I,ref(L))          & space + jmpd \\
ret                     & return (when no environment) \\
retd                    & return (when no environment, no choicepoint) \\
retn                    & return (when no environment, with choicepoint) \\
\hline
\end{tabular}

Calls via label are currently only used for recursive calls, in order to
make it possible to recompile individual predicates without having to
re-link the corresponding calls.

A number of common instruction combinations are supported as well:
\index{chain instruction}
\index{exit instruction}

\begin{tabular}{|p{0.35\textwidth}|p{.6\textwidth}|}
\hline
chain(ref(L))           & deallocate + jmp \\
chain(P)                & deallocate + jmp \\
chainc(ref(L))          & cut + deallocate + jmp \\
chainc(P)               & cut + deallocate + jmp \\
chaind(ref(L))          & deallocate + jmpd \\
chaind(P)               & deallocate + jmpd \\
exit                    & deallocate + ret \\
exitd                   & deallocate + retd \\
exitc                   & cut + deallocate + ret \\
\hline
\end{tabular}



\subsubsection{Other Control Transfer}
In addition, there are simple inter-procedural control transfer instructions.
They are used to skip the read-mode sequence in nested unification code, or
to skip alternatives of disjunctions.
\index{branch instruction}

\begin{tabular}{|p{0.35\textwidth}|p{.6\textwidth}|}
\hline
branch(ref(L))          & intra-procedural control transfer \\
branchs(I,L)            & branch L + space I \\
\hline
\end{tabular}


\subsubsection{Environment and Local Stack Temporaries}
Instructions for allocating, deallocating and initialising environment frames
on the local stack. Note that there are compound instructions that include
allocation or deallocation functionality. Also, temporaries are often
pushed onto the local stack as a side effect of other instructions, rather
than by calling space(I).
\index{allocate instruction}
\index{deallocate instruction}
\index{initialize instruction}
\index{space instruction}

\begin{tabular}{|p{0.35\textwidth}|p{.6\textwidth}|}
\hline
allocate(N)             & allocate an environment \\
deallocate              & deallocate topmost environment \\
initialize(y(VList))    & initialise several environment slots with free variables \\
initialize_named(y(NVList))     & initialise several environment slots with named variables \\
space(I)                & adjust local stack pointer \\
\hline
\end{tabular}

\subsubsection{Cut}
The cut instructions save pointers to choice points, or pop the choice point
stack up to a previously stored position. Note that {\eclipse} does not
clean up the trail stack on this occasion (although removing choice points
can render trail entries redundant), but leaves this to the garbage collector.
Note that there are compound instructions (chainc, exitc) which contain
cut functionality.
\index{cut instruction}
\index{savecut instruction}

%\begin{tabular}{|l|l|}
\begin{tabular}{|p{0.35\textwidth}|p{.6\textwidth}|}
\hline
savecut         & save cut point in y(1) \\
savecut(a(A))   & save cut point in a(A) \\
savecut(y(Y))   & save cut point in y(Y) \\
neckcut         & pop topmost choicepoint \\
cut(N)          & cut to point in y(1), trim environment to N slots \\
cut(y(Y),N)     & cut to point in y(Y), trim environment to N slots \\
cut(a(A))       & cut to point in a(A) \\
softcut(y(Y))   & disable (deep) choicepoint referred to by y(Y) \\
cut_single      & cut to point in y(1) iff there is exactly one choicepoint to cut \\
\hline
\end{tabular}


\subsubsection{Checkpoints}
The purpose of these instructions is to insert additional points for global
stack overflow checks, or event handling (e.g.\ waking).
\index{res instruction}
\index{gc_test instruction}

%\begin{tabular}{|l|p{10cm}|}
\begin{tabular}{|p{0.35\textwidth}|p{.6\textwidth}|}
\hline
res(Na,Ne)      & handle events if any \\
ress(Nt,Na,Ne)  & space Nt + res Na,Ne \\
gc_test(M)      & make sure M pwords can be pushed onto global stack
                 (expand if necessary, but don't garbage collect)\\
\hline
\end{tabular}

\subsubsection{External Predicate invocation}
External predicates\index{external predicates}  are predicate implemented in the implementation
language, or via the embedding interface.  Their procedure descriptor
holds their function address, and they are invoked from the abstract
machine using dedicated external-instructions.
Arguments are passed in argument registers, as for regular Prolog
predicates.
\index{external instruction}

A number of built-ins are implemented within the emulator and
invoked via the escape instruction, passing arguments via the
local stack\footnote{it is intended to drop this calling convention
in release 6}.
\index{escape instruction}

%\begin{tabular}{|l|l|}
\begin{tabular}{|p{0.35\textwidth}|p{.6\textwidth}|}
\hline
external(P,Addr)        & invoke a C function at Addr \\
external0(P,Addr)        & special case arity 0 \\
external1(P,Addr)        & special case arity 1 \\
external2(P,Addr)        & special case arity 2 \\
external3(P,Addr)        & special case arity 3 \\
escape(P)               & execute one of the built-ins
                        that are implemented inside the emulator function\\
\hline
\end{tabular}

\subsubsection{Metacalls}
See also section \ref{secmetacall}). These instructions are generated by
the compiler or used in handcoded sequences to support metacalling
(the invocation of non-compiled goals represented as runtime data structures).
\index{metacall}
\index{metacall instruction}

%\begin{tabular}{|l|l|}
\begin{tabular}{|p{0.35\textwidth}|p{.6\textwidth}|}
\hline
metacall(N)     & for the call/1 and @/2 built-in \\
meta_jmp        & for the call/1 and @/2 built-in \\
explicit_jmp    & for the :/2 built-in \\
\hline
\end{tabular}

\subsubsection{Debugging}
Debug instructions are inserted by the compiler when compiling code in
debug mode.  They call-port generating ones are prefixed to the normal
calling instructions.  The exit port generating instruction is part of
a code sequence which gets dynamically inserted into the success
continuation of a traced predicate.
\index{debug instruction}

%\begin{tabular}{|l|l|}
\begin{tabular}{|p{0.35\textwidth}|p{.6\textwidth}|}
\hline
debug_call(P,Port)      & generates tracer port for regular call \\
debug_esc(P,Port)      & generates tracer port for external call \\
debug_exit              & generates tracer EXIT port \\
\hline
\end{tabular}

\subsubsection{Special purpose instructions}
These are only used in handcoded code sequences:
\index{catch instruction}
\index{throw instruction}
\index{fastcall instruction}
\index{handler_call instruction}
\index{undefined instruction}
\index{continue_after instruction}
\index{suspension_call instruction}
\index{wake instruction}
\index{nop instruction}
\index{exit_emulator instruction}

%\begin{tabular}{|l|p{10cm}|}
\begin{tabular}{|p{0.35\textwidth}|p{.6\textwidth}|}
\hline
catch           & for the block/3 (catch/3) built-in predicate \\
throw           & for the exit_block/1 (throw/1) built-in predicate \\
\hline
fastcall(I,N)           & invoke handler for exception I \\
handler_call(I,N)       & invoke handler for interrupt I \\
undefined(P)            & raise UNDEFINED exception for predicate P \\
continue_after_exception& restore state after exception \\
\hline
continue_after_event            & restore after synchronous event \\
continue_after_event_debug      & \\
\hline
suspension_call(N)      & for call_suspension/1 \\
suspension_jmp          & for call_suspension/1 \\
\hline
wake_init(N)            & for the wake/0 built-in \\
wake                    & for the wake/0 built-in \\
\hline
nop                     & no operation \\
\hline
ret_nowake              & ret without event check \\
retd_nowake             & retd without event check \\
exitd_nowake            & exitd without event check \\
\hline
exit_emulator(I)        & exit emulator with return code I \\
\hline
\end{tabular}


%----------------------------------------------------------------------
\section{Procedure Call and Return}
\label{seccallret}
%----------------------------------------------------------------------

Regular predicates are called by
\index{argument passing}
\begin{enumerate}
\item loading their arguments into the
first N argument registers A1..An, where N is the arity of the callee.
This is usually achieved by a sequence of instructions of the Put family,
\index{put instruction}
but can also be done e.g.\ by the generic metacall code (see \ref{secmetacall}).
\item transferring control by either a {\bf Call} instruction or
a {\bf Jmp/Chain} instruction. Call instructions push a return address\index{return address}
(CP in the diagrams) onto the global stack and set PP to the first
instruction of the called predicate's code.
Jmp/Chain instructions only set PP (they are used for tail calls and
don't return). Chain instructions in addition deallocate an environment.
\end{enumerate}
\begin{figure}
\epsfbox{localframes.eps}
\label{localframes}
\caption{{\eclipse} local stack frames}
\end{figure}
A Call instruction requires that the caller has allocated an
environment\index{environment}  (see figure \ref{localframes}).
The last word of the Call instruction is an environment
size, i.e.\ the number of environment slots that are still needed
during the execution of the subgoal (see figure \ref{callinstr}). 
\begin{figure}
\epsfbox{callinstr.eps}
\label{callinstr}
\caption{Call instruction, return address and environment size}
\end{figure}
Environment slots are sorted according to their lifetimes\index{lifetimes}, so that the
ones that become useless first have the highest numbers.  This is a
\index{trimming} form of {\bf environment trimming}, although not the
classical one:  the environments are not physically shortened, and the
variables that have their last occurrence in the current subgoal are
still considered active.
The environment size field in the call instructions is used by the
garbage collector only, in order to know which of the environment
slots should still be considered active.

The abstract machine has a DET\index{DET}  flag, which is always set when a procedure
is entered, and cleared when a choicepoint is created. It is used by
\begin{itemize}
\item 
the neckcut\index{neckcut}  instruction to detect whether the procedure has created
a choicepoint that need to be cut.
\item 
the savecut instruction to detect which choicepoint address to save.
\item 
the ret and tail call instructions to know whether to pop the return
address from the local stack or not.
\end{itemize}



%----------------------------------------------------------------------
\section{Nondeterminism}
\label{secnondet}
%----------------------------------------------------------------------
Nondeterminism\index{Nondeterminism}  relies on choicepoints\index{choicepoints}  and trailing\index{trailing}  to reset the machine
to an earlier state.  The layout of choicepoints is depicted in figure
\ref{figchp}.
\subsection{Choicepoints}
\begin{figure}
\epsfbox{controlframes.eps}
\caption{{\eclipse} control stack frames}
\label{figchp}
\end{figure}
A number of variants of the standard choicepoint exist that store more
state (for exceptions\index{exceptions}  and invoking a recursive engine\index{engine}), or less (small
choicepoints, only BP and PB field, for shallow backtracking\index{shallow backtracking}).

\index{try instruction}
The choice point creation instructions (try) simply store a subset of the
abstract machine registers in the choicepoint frame, and set GB and EB
registers, which will lead to future trailing of any modification to data
older than the new choicepoint.

\index{retry instruction}
\index{trust instruction}
The retry/trust instructions restore this information and transfer control
to the alternative clause or disjunctive branch (the trust instruction
also discard the choicepoint). In addition, untrailing is performed, i.e.\
all changes recorded since choicepoint creation (or restoration) are
undone.  A number of abstract machine registers are reinitialised
(EB\index{EB}, GB\index{GB}, GCTG\index{GCTG}, TG_SL\index{TG_SL}, MU\index{MU}, DE\index{DE}). 
Furthermore, these instructions include
unconditional hooks for the debugger\index{debugger}, because it is possible that
execution fails out of a traced portion of the execution, and we want
to trace this event even if the failure continuation is within code
that has been compiled without tracing support.


\subsection{Timestamps}
\label{sectimestamps}
For some purposes, e.g.\ the avoidance of unnecessary trailing\index{trailing!unnecessary},
we need unique identifiers for choicepoints. Choicepoint addresses
cannot serve this purpose, since choicepoints may be cut or popped,
and new ones re-created at the same address.
We therefore would like to use global stack addresses as timestamps\index{timestamps},
namely the value the stack pointer had when the choicepoint was created.
To make this safe, we must guarantee that there is always something
pushed on the global stack between two choicepoints, otherwise timestamps
could become identical when they really belong to different choicepoints.

Our solution is to push a "witness" word with every choicepoint. 
Their addresses are used as the time stamps.
The GB\index{GB}  register always points to the current witness\index{witness}.
A stamp looks like a [] (a ref to a TNIL\index{TNIL}  of the proper age).

An advantage compared with other timestamping schemes is that the
witness (ie. the timestamp value) can be garbage collected once
the choicepoint and all timestamps have disappeared.
Compared to the scheme used in setarg, we hopefully use less memory
since an extra word is pushed per choicepoint rather than per trailed
modification.
Unfortunately, the timestamps don't collapse once the choicepoint between
them is cut.

Locations that need timestamped modifications need to have space for
a pointer.


%----------------------------------------------------------------------
\section{Trailing}
%----------------------------------------------------------------------
\subsection{Binding and Value Trail}
\index{trailing}
The trail stack records modifications to memory locations which must be
undone on backtracking. There are two forms of trail for variable bindings
(with and without tag), and a general value trail\index{value trail}, see figure
\ref{figtrailframes}.
\begin{figure}
\epsfbox{trailframes.eps}
\caption{{\eclipse} trail stack frames}
\label{figtrailframes}
\end{figure}
The tagged binding trail is needed to trail bindings to non-standard
variables, e.g.\ attributed variables\index{attributed variables}.
The value trail is used in particular for setarg/3\index{setarg/3}.


\subsection{Undo Trail}
A more general undo mechanism is implemented by the undo trail\index{undo trail}  frames.
This is used e.g.\ in interfacing\index{interfacing}  code to {\eclipse} that is not backtracking
aware. We support a simple form and a timestamped form. Timestamping
is a mechanism to avoid multiple trails of modifications to the same
object, when there was no choicepoint in between modifications.
\index{timestamps}

Simple and stamped undo frame
(see \ref{figundotrail})
differ only in that the stamped frame has two extra fields, the stamp
address and the old stamp value.
Both frames have a pointer to an "item" which is the related data
structure that determines the lifetime of the frame.
\begin{figure}
\hfill
\begin{minipage}[t]{.45\textwidth}
\begin{tiny}
\begin{verbatim}
+--------+--------+--------+--------+    \
|               data                |     |
.                                   .     |
.                                   .      > size + 1 32 bit words
.                                   .     |
|               data                |     |
+--------+--------+--------+--------+    /
|         & untrail_function        |
+--------+--------+--------+--------+
|           item address            |
+--------+--------+--------+--------+
|            frame size    |0000tt11|
+--------+--------+--------+--------+   <-- TT
\end{verbatim}
\end{tiny}
\end{minipage}
\hfill
\begin{minipage}[t]{.45\textwidth}
\begin{tiny}
\begin{verbatim}
+--------+--------+--------+--------+    \
|               data                |     |
.                                   .     |
.                                   .      > size + 1 32 bit words
.                                   .     |
|               data                |     |
+--------+--------+--------+--------+    /
|          old stamp value          |
+--------+--------+--------+--------+
|           stamp address           |
+--------+--------+--------+--------+
|         & untrail_function        |
+--------+--------+--------+--------+
|           item address            |
+--------+--------+--------+--------+
|            frame size    |0001tt11|
+--------+--------+--------+--------+   <-- TT
\end{verbatim}
\end{tiny}
\end{minipage}
\hfill
\caption{Simple and timestamped undo trail frame}
\label{figundotrail}
\end{figure}
The tt field indicates the type of trailed data, as in the value trail.
The frames are created by a single function:
\begin{verbatim}
ec_trail_undo(&function, &item, &stamp,
                        &data, data_size_in_words, trail_flags)
\end{verbatim}
The arguments are
\begin{description}
\item[function] 
    address of the undo function
\item[item] 
    address of the data structure to which the undo applies.
                This can be a global stack address (in which case it must
                point to a properly tagged pword) or an arbitrary heap
                address (in which case the content does not matter), or
                it can be NULL. If item is a global stack item, and this item
                is about to be garbage collected, the undo function will be
                called early. If item is NULL or outside the global stack,
                the undo function will only be called on failure.

\item[stamp] 
    address of a timestamp (see \ref{sectimestamps}).
    This should be NULL if you don't want
                to use timestamping. Using timestamping implies that multiple
                trails within the same choicepoint segment are redundant, and
                the system will suppress all but the first. The location of
                the timestamp is immaterial. It can be part of the item or not.
                It will be kept alive by the existence of the trail frame.
\item[data,size] 
    address and size (in words) of data to be passed to the
                undo function. This data will be copied into the trail frame.
\item[trail_flags]      the type of the data (TRAILED_PWORD or TRAILED_WORD)
\end{description}
The undo function is called as follows:
\begin{verbatim}
undo(pword *item, word *data, int data_size_in_words, int undo_context)
\end{verbatim}
Untrailing is done in
    undo_context == UNDO_FAIL\index{UNDO_FAIL}
        on failure (unless redundant according to stamp), or in 
    undo_context == UNDO_GC\index{UNDO_GC}
        on gc, when the untrail can be done early (when the item is
        inaccessible in the post-trailed state).

If the data consists of pwords, early untrailing during garbage
collection is tricky:  the data may be marked (and thus unusable)
at the time the untrail function is called. If the undo function
needs to look at the data even for redundant untrails, we must make
sure at trail time that the data that is referenced by the trail
frame is not referenced by anybody else.

CAUTION1: The data fields are blindly copied and when the untrail
function is executed, the data may not be properly aligned if they
require >4 byte boundary alignment\index{alignment}  (e.g. doubles). Accessing these
misaligned field directly can cause a segmentation violation on some
architectures. In these cases, the data should be copied (using memcpy()) 
to a properly aligned data structure first.

CAUTION2:  The GC does NOT mark the data pointed to by the item
address field.  This item is only used to determine the lifetime
of the trail frame.  In particular, this undo-trail CANNOT be used
to implement a general value-trail because it does not assume
(during GC) that the item has been modified at trailing time, so
the current value would not be marked and updated properly during
GC.  It is in principle a pure side-effect trail.  However, simple
modifications in the item, like resetting counters or bits, should
be ok.

Another difference between a reset-trail and side-effect trail is that
the reset-trail is redundant when the item is about to disappear anyway
during failure. The side effect still needs to be done in that case
(e.g. cleanup/deallocation for external handles\index{handles}).


%----------------------------------------------------------------------
\section{Destructive Assignment}

\index{setarg/3}
The builtin setarg/3 and several other destructive assignment operations
use the single C function \verb.ec_assign(pword *arg, value v, tag t).
\index{ec_assign}
which backtrackably overwrites the location arg with the new value v/t.
The code uses the macros \verb.NewLocation(pword*).
which means the address is younger than the most recent choicepoint\index{choicepoint},
and \verb.NewValue(value,tag).
which means the pword is younger than the most recent choicepoint
(when it is a constant, it is assumed to be old).

Assume the location to be assigned to is Arg, its old value Old and
its new value New.  Then there are the following different cases.

\begin{enumerate}
\item NewLocation(Arg) - 
    Simply overwrite without trailing, regardless of old or new value,
    because the everything will disappear an backtracking.
\item !NewLocation(Arg), NewValue(New) and NewValue(Old)  -
    Simply overwrite without trailing,
    because the old value is unaccessible after backtracking anyway.
\item !NewLocation(Arg, NewValue(New) and !NewValue(Old) -
    overwrite with trailing.
\item !NewLocation(Arg) and !NewValue(New) -
Copy new value to top of stack, and assign a reference to it,
   instead of the new value itself (like 2).
    This is done so that the next time an assignment is attempted, the
    (then old) value's address indicates how old the assignment is,
    and trailing can hopefully be avoided.  This trick avoids the need
    for an explicit timestamp.
\end{enumerate}
The actual ec_assign() function also checks for the New value being
a variable on the local stack. Since references from global to local
are not allowed, the variable gets first globalised\index{globalised}  and the result
is assigned instead (leading then to case 1, 2 or 3).



%----------------------------------------------------------------------
\section{Attributed Variables}

\index{attributed variables}
Attributed Variables are described in \cite{meier92} and
\cite{metaterms95}, where they are referred to as
metaterms\index{metaterms} - some of the {\eclipse} documentation and
implementation uses these terms interchangeably.  One can think of
them as a variable with (one or more) attached attributes.  In
{\eclipse} syntax we can write e.g. X\{Attribute\}.  Attributed
variables behave in some respects like variables, but in most respects
their behaviour is user-definable (e.g.  what happens with them on
unification\index{unification}, copy_term\index{copy_term} etc). 
Also, they can be decomposed into their components (variable and
attributes) via matching clauses, emphasising their meta-level aspect.

The idea that something like "attributed variables" is needed for the
implementation of coroutining\index{coroutining}  Prologs and CLP\index{CLP}  languages is rather obvious
and has been used in implementations long before anybody invented a
special name for it. However, they were usually not accessible as such
for the Prolog/CLP programmer.
We can credit Ulrich Neumerkel and Christian Holzbaur for suggesting
that adding the metaterm primitive to a Prolog machine is in principle
all that's needed to build constraint solvers on top.

In {\eclipse} we have gone a step further and made an effort to fully
integrate attributed variables into our language as first class citizens.
I.e.\ they have their own syntax, they are integrated in unification
and indexing of the abstract machine, they are handled appropriately
by almost all builtin predicates, etc. Another important point is that
{\eclipse} attributed variables can have multiple independent
attributes, which makes it possible to use several constraint solvers
at the same time, for example.  The variety of {\eclipse}'s constraint\index{constraint}
solver libraries demonstrates this.



%----------------------------------------------------------------------
\section{Metacalls}
\label{secmetacall}
%----------------------------------------------------------------------

\index{metacall}
Basically, metacalling is very simple:  Given a structure, load its
arguments into the argument\index{argument}  registers and jump to the predicate code
specified by the structure's functor.  The actual implementation is
more complex because of handling the cut, modules, and different call
protocols for externals and prolog predicates.

The central piece is the Metacall/Metajmp instruction.
It expects the argument registers to be loaded as follows:
\begin{description}
\item[A1] the goal structure to be metacalled
\item[A2] the context module
\item[A3] the lookup module
\item[A4] the cut pointer
\end{description}
\index{"@/3}
\index{"@/2}
Apart from the cut\index{cut}  pointer, these are the arguments of @/3, which is
the tool body (see User Manual chapter on the module system) of @/2.
The abstract code sequence of @/3 is simply the following:
\begin{verbatim}
@/3:
        SavecutAM       A4
        Meta_jmp
\end{verbatim}
and the code of call/2 (the tool body of call/1) is
\begin{verbatim}
call/2:
        Move            A2 A3
        SavecutAM       A4
        Meta_jmp
\end{verbatim}
In order to handle cuts inside the metacalled goal, the value of the B
register at the beginning of the metacall is loaded into an argument
and passed to the instruction.

The Metacall/Metajmp instruction first does the necessary dereferencing
and type checks on arguments 1 and 3.
Then the visible predicate is found by calling
the procedure table lookup function visible_procedure\index{visible_procedure}().

The next point is to check for goals that must be handled in a special
way because they are defined as being transparent to cuts.
These are conjuction, disjunction, implication, if-then-else\index{if-then-else}  and cut.
They are all translated into a predicate that takes an additional
argument which is the cut pointer. This cut pointer is the value of the
B stack pointer at the beginning of the metacall.
\begin{verbatim}
Goal in Module                  Translated goal
--------------                  ---------------
Goal1 , Goal2                   ','(Goal1, Goal2, Module, Cut)
Goal1 ; Goal2                   ';'(Goal1, Goal2, Module, Cut)
Goal1 -> Goal2                  '->'(Goal1, Goal2, Module, Cut)
Goal1 -> Goal2 ; Goal3          ';'(Goal1, Goal2, Module, Cut, Goal3)
!                               cut_to(Cut)
\end{verbatim}
The transformed predicates could be defined as follows (although
in reality they are implemented by abstract code sequences in code.c):
\begin{verbatim}
    ','(Goal1, Goal2, Module, Cut) :-
            call(Goal1, Module, Module, Cut),
            call(Goal2, Module, Module, Cut).

    '->'(Goal1, Goal2, Module, Cut) :-
            call(Goal1, Module, Module, []).
            !,
            call(Goal2, Module, Module, Cut).

    ;(Goal1, Goal2, Module, Cut) :-
            call(Goal1, Module, Module, Cut).
    ;(Goal1, Goal2, Module, Cut) :-
            call(Goal2, Module, Module, Cut).

    ;(Goal1, Goal2, Module, Cut, Goal3) :-
            call(Goal1, Module, Module, []).
            !,
            call(Goal2, Module, Module, Cut).
    ;(Goal1, Goal2, Module, Cut, Goal3) :-
            call(Goal3, Module, Module, Cut).
\end{verbatim}
We have written the Metacall/Metajmp instruction as call/4.
Note also that the Cut pointer is not passed into the conditions
of implication and if-then-else: these are not transparent to the
cut, as this could interfere with the subsequent cut of the condition
itself.

After this goal transformation, the goal arguments are moved to the
appropriate locations, i.e. the argument registers (when the goal is a
regular Prolog procedure) or dereferenced and pushed on the local
stack (when the goal is an external).  When the goal is a tool
interface, the module argument is also added.  The last step is to
actually jump to the code of the prolog goal or to call the external,
respectively.



%----------------------------------------------------------------------
\section{Structure unification}
\label{secstructunify}
%----------------------------------------------------------------------
\index{structure unification}
\index{compound term compilation}
Structure unification is compiled differently from the WAM, and is
quite involved. The information in this section is intended as
supplement to Micha's paper on the issue \cite{compnd}.

\subsection{Head unification}

\index{read sequence}
\index{write sequence}
\index{get instruction}
The basic scheme is to have separate read- and write-sequences,
for deconstructing/unifying an existing structure, and for constructing
a new structure. The get_structure type instructions dispatch according
to the instantiation of their argument:
\begin{verbatim}
        get_structure a(A) F ref(Lr)
        write_...
        ...
        write_...
        branch Le
Lr:
        read_...
        ...
        read_...
Le:
\end{verbatim}
For unification of nested structures, it may be necessary to jump back
and forth between read and write sequences, depending on whether substructures
are present in the runtime term or not.

   The basic idea is to unify nested compound terms top-down and
   left-to-right. Unlike the WAM scheme, this method does not require
   temporaries to hold structure arguments, but needs a stack instead.
   However, since the depth of the nested term in the head is known
   at compile time, this stack can be built from temporaries (every
   nesting level is assigned one temporary\index{temporary}, except the bottom level).
   These temporaries contain a read/write mode flag and a copy of the
   S register, indicating how and where to continue after having
   finished the unification of a compound subterm.
   This method is better than the WAM scheme especially for wide,
   flat structures and for right-balanced structures like lists.

   Read and write-mode are in separate code sequences, and there
   are conditional jumps back and forth between the sequences.
   If a read-instruction discovers a variable in the input, it
   creates a structure frame and jumps into the write-sequence to
   construct the structure arguments. The 'return address' in form
   of a read-flag and the next value of S is saved in a temporary.
   At the end of a write sequence for all arguments of a subterm,
   the temporary is tested and possibly control is transferred back
   to the read mode. This is all further complicated by a 'last-call'
   optimisation, ie. dropping the temporary before the last subterm.

   Compared to the presentation in Micha's paper, in the actual
   implementation instructions are merged and specialised:
\begin{verbatim}
Write mode:                     Read mode:

First                           (part of Read_structure WLabel)
    allocate Ti                     allocate Ti
    down (save S+1|WRITE)           down (save S+1|READ)
                                      possibly goto write mode

Next Ti RLabel                  (part of Read_next_struct Ti WLabel)
    possibly goto read mode         up (restore S)
    up (restore S)                  down (save S+1)
    down (save S+1)                 possibly goto write mode
                      
Mode Ti RLabel                  Mode Ti
    up (restore S)                  up (restore S)
    possibly goto read mode

Next Ti                         (part of Read_structure Ti WLabel)
    down (save S+1)                 down (save S+1)
                                    possibly goto write mode
\end{verbatim}


\subsection{Body subgoal arguments}

\index{put instruction}
    The terms are built breadth-first, top-down, using two pointers.
    TG is the allocation pointer and S is the write-pointer, lagging
    behind and filling the allocated space.



%----------------------------------------------------------------------
\section{Exceptions}
%----------------------------------------------------------------------

\index{exception}
\index{block/3}
\index{exit_block/1}
Exception handling corresponds to the builtins block/3 and exit_block/1.
block/3 is a tool and block/4 is its tool body.
They are implemented via the following handcoded abstract instruction
sequences:
\begin{verbatim}
block/4:             // block(Goal, Catcher, Recovery, Module)
        Catch                           // special instruction
        Allocate        1
        Savecut
        Move            A2 A3           // call(Goal)
        Savecut         A4
        Metacall        1
        Cut_single      0               // clean up catch frame
        Exit

exit_block/1:   % exit_block(Ball)
        Throw                           // special instruction
        Move            A2 A3           // call(Recovery)
        Savecut         A4
        Meta_jmp
\end{verbatim}
These use special-purpose instructions:
\begin{description}
\item[Catch] 
    \begin{itemize}
    \item checks the Catcher argument in A[2] for simple type or variable
    \item moves the module argument from A[4] to A[2] (for subsequent call/2)
    \item builds a catch frame on the control stack, containing:
                sp, tg, tt, e, Catcher, Recovery, Module
    \end{itemize}
\item[Throw] 
    \begin{itemize}
    \item check the "Ball" argument in A[1] for simple type or variable
    \item pop frames off the control stack until a catch frame is found, whose
          Catcher entry would unify with Ball.
          If an invocation frame is encountered while popping, we have to exit
          an emulator invocation and continue executing the BIThrow in the
          previous emulator.
    \item If the corresponding catch frame is found:
        \begin{itemize}
        \item restore sp, tg, e from catch frame, untrail
        \item pop the catch frame
        \item reset EB, GB from the choicepoint below the catch frame
        \item pop the catch frame
        \item load A[1] with the Recovery goal, A[2] with the Module
          (for subsequent call/2)
        \item unify Catcher and Ball
        \end{itemize}
    \end{itemize}
\item[Cut_single] 
    Will cut the catch frame if it is on top of the stack, i.e.\ if the
    goal has not left any choicepoints itself.
\end{description}
We guarantee that a catch frame is always found by enclosing
the toplevel loop with
\begin{verbatim}
    block(loop, Tag, notag(Tag))
\end{verbatim}
This serves as a "catch-all" for exit_block's.

The corresponding ISO Prolog primitives catch/3 and throw/1 are similar
but allow complex terms to be thrown. {\eclipse} should eventually migrate
to support that (but preferably without full heap copying).



%----------------------------------------------------------------------
\section{Suspension Mechanism}
%----------------------------------------------------------------------

{\eclipse} can deal with delayed goals, i.e. goals that are part of the
resolvent\index{resolvent}, but are not part of the current continuation. They will only
be reactivated by certain wakeup events. Once reactivated, they enter
a priority-based scheduler, and are executed as soon as there are no
higher-priority goals scheduled (see priority mechanism \ref{chappriority}).

Delayed goals can be in 3 states, SLEEPING\index{SLEEPING}, SCHEDULED\index{SCHEDULED}  and DEAD\index{DEAD}.
The transitions are depicted in figure \ref{figsuspstates}.
\begin{figure}
\hfill
\begin{minipage}[t]{.45\textwidth}
\begin{tiny}
\begin{verbatim}
Non-demon:
                schedule
       SLEEPING --------> SCHEDULED
           |                 |
      kill |                 | execute/kill
           \                 /
            ----> DEAD <-----
\end{verbatim}
\end{tiny}
\end{minipage}
\hfill
\begin{minipage}[t]{.45\textwidth}
\begin{tiny}
\begin{verbatim}
Demon:          schedule
                -------->
       SLEEPING           SCHEDULED
           |    <--------    |
           |     execute     |
      kill |                 | kill
           \                 /
            ----> DEAD <-----
\end{verbatim}
\end{tiny}
\end{minipage}
\hfill
\caption{States in the life of a suspension}
\label{figsuspstates}
\end{figure}
Every delayed goal is represented by a so-called suspension
\index{suspension}
\index{delayed goal}
which is created by the make_suspension/3,4 builtin.
Immediately after creation, the suspension is in the SLEEPING state.
Two transitions are possible from this state, towards the SCHEDULED state
via the schedule_suspension/1,2 builtin, or towards the DEAD state via
the kill_suspension/1 builtin.
\index{make_suspension/3}
\index{schedule_suspension/2}
\index{kill_suspension/1}

Once scheduled, the goal will eventually be ready for execution, as
soon as all higher-priority goals, and all previous goals in the queue for
the same priority are finished executing.  Then, just before execution starts,
the state changes to either of DEAD (for standard predicates), or back
to SLEEPING (for predicates with the demon-property, see demon/1 declaration).
For demons, the only way to reach the DEAD state is to be killed explicitly
via kill_suspension/1.


\subsection{Suspension Descriptor}

A delayed goal is represented by a TSUSP\index{TSUSP}-tagged pointer
to a suspension descriptor on the global stack (see figure \ref{figsuspension}).
This descriptor is created by the make_suspension/3,4
built-in (implemented within the emulator).
\begin{figure}
\hfill
\begin{minipage}[t]{.9\textwidth}
\begin{tiny}
\begin{verbatim}
|-----------------|
|                 |
|- - - MODULE  - -|
|                 |
|-----------------|
|                 |
|- - -  GOAL - - -|
|                 |
|-----------------|
|     PRIO  S TREF| <= these are mutable fields
|- - - STATE - - -|
|    timestamp    |
|-----------------|                                  |-----------------|
|      INVOC      |     <= CAUTION: no tag!          |      INVOC      |     <= CAUTION: no tag!
|- - - - - - - - -|                                  |- - - - - - - - -|
|       PRI       |                                  |       PRI       |
|-----------------|       |---------|                |-----------------|       |---------|
|0--       XD TDE |       |  TSUSP  |                |0--       1D TDE |       |  TSUSP  |
|- - - - - - - - -|       |- - - - -|                |- - - - - - - - -|       |- - - - -|
|       LD        |    /-------     |                |      NULL       |    /-------     |
|-----------------|<--/   |---------|                |-----------------|<--/   |---------|
\end{verbatim}
\end{tiny}
\end{minipage}
\hfill
\caption{The suspension descriptor (live and dead)}
\label{figsuspension}
\end{figure}
The fields are:
\index{LD}
\index{TDE}
\begin{description}
\item[LD] used to chain all live suspensions together. The chain starts
with the LD abstract machine register. This list is used to implement the
built-ins suspensions/1, current_suspension/1, delayed_goals/1 and subcall/2.
\item[D (demon-flag)] indicates whether the goal is a demon and can be
re-suspended.
\item[X (dead-flag)] indicates that the goal is dead.
\item[TDE] The tag marking the suspension descriptor (for the garbage collector).
\item[PRI] pointer to the goal's procedure descriptor, to speed up the
waking process.
\item[INVOC] debugger invocation number.
\item[S (scheduled-flag)] indicates that the goal has been scheduled for
execution, but has not yet been executed. While set, the suspension is in the
waking list.
\item[timestamp] used to optimize trailing of the scheduled-flag.
\item[PRIO] goal waking priority.
\item[GOAL] the goal as a compound term (or atom).
\item[MODULE] context module of the delayed goal.
\end{description}
When the suspension is dead, then the descriptor may be partially
garbage collected, i.e. goal and module may no longer be present and
it may be removed from the LD list.


\subsection{Suspension Lists}
\index{suspension list}
Suspensions are usually entered into lists that are an argument of a structure,
typically one argument of an attribute structure.
The builtins
init_suspension_list/2,
enter_suspension_list/3,
merge_suspension_lists/4
and 
insert_suspension/4
are dealing with this. There are no explicit removal operations for suspensions
in these lists. however, dead suspensions are removed opportunistically
when a lists is traversed for the purpose of scheduling, and dead
suspensions at the head of the list are removed when new suspensions are
prefixed.


\subsection{Waking scheduler and priority system}
\label{chappriority}

\index{wake}
\index{call_priority/2}
{\eclipse} currently supports a system of 12 fixed priorities.
Initially, a program runs at priority\index{priority}  12. The running priority can be raised
either explicitly by call_priority/2, or it is raised implicitly when
a higher priority goal interrupts the execution of a lower-priority one.
Delayed goals have a waking-priority attached, which is either derived
from the predicate's priority setting, specified explicitly when the
suspension is created, or can be modified via set_suspensions_data/3.

The scheduler for suspensions consists of:
\begin{itemize}
\item An array of woken suspension lists, one for each priority.  Each
        scheduled goal gets inserted into the list according to its
        priority by schedule_suspensions/2.
        Once a suspension is scheduled into a list, its
        priority cannot be changed any longer.

\item The scheduling loop implemented by the builtin wake/0 (Wake instruction),
        \index{wake instruction}
        which is usually invoked implicitly, e.g. by
        attributed variable unification handlers.  This loop scans the
        array of woken lists, and executes the corresponding goals one
        by one, in order of priority.
\end{itemize}
The scheduler makes sure each goal executes `at its own priority' by setting
the machine's WP\index{WP}  register accordingly. This makes sure that only higher
priority goals can wake up inside the just woken one.

%----------------------------------------------------------------------
\section{Event Handling}
\label{seceventhandling}
%----------------------------------------------------------------------
\index{event}
{\eclipse} has a notion of synchronous\index{synchronous event}  events, which can be posted to
the abstract machine.  There is a queue for posted events, they can be
either atoms (symbolic events), integers (for synchronous signal
handling), or event handles\index{handle!event}  (anonymous events).  Their corresponding
handlers are then executed as soon as the engine reaches a trigger
point.  Such points are those where the abstract machine is in a well
known state:
\begin{itemize}
\item CALL: When a (regular) predicate is about to be called (at the end of
a Call, Chain or Jmp instruction): PP points to start of procedure, return
address on top of local stack, arguments registers hold arity arguments.
\item Return: When a (regular) predicate is about to return:
No argument registers are valid, return address just popped from stack.
\item Res: At an explicit Res (resume) instruction: return address on top
of local stack, arity argument registers valid.
\end{itemize}
In order to minimise the overhead of the event test in the abstract machine, 
the event handling mechanism is triggered in the same way as a stack
garbage collection. Stack overflow is triggered by the global stack pointer
(TG\index{TG}) growing beyond a limit register (TG_SL\index{TG_SL}).  For event handling, such an
overflow is simulated by manipulating the limit register.
When the machine detects such an overflow, it first checks for a true stack
overflow, if this is not the case, the MU\index{MU}  register is checked for a
attributed variable (metaterm) unification\index{unification}  event, and if that is not set,
the general event queue is checked.

The event handler itself is either looked up in a table (for numeric events),
as a property of an atom (for symbolic events), or copied from a heap term
(for anonymous events). A corresponding goal is then constructed and called.


%----------------------------------------------------------------------
\section{Abstract Machine Emulator}
%----------------------------------------------------------------------

The abstract machine\index{abstract machine}  emulator\index{emulator}  is at the heart of \eclipse's runtime system. 
It is written in C with optional use of GCC features to implement
threaded code.

The emulator is quite a large (but flat) piece of code and consists of the single
function ec_emulate().  The decision to have it as a single function
is due to technical reasons, especially the need to use register
variables for efficiency.  If it were distributed over several
functions, the state of the abstract machine would have to be stored
in global variables or passed as arguments.

The main part of the emulator is an endless loop (in the non-threaded
version) that reads one instruction code from the location the program
pointer (PP) points to, and executes a switch to go to the piece of
code that implements the instruction.
\begin{small}
\begin{verbatim}
while(1)
{

    switch((pp++)->inst)
    {
        case MoveAM:
            <code for instruction>
            Next-Pp;

        case MoveAMAN:
            <code for instruction>
            Next-Pp;

        ...

        default:
            <undefined instruction>
    }
}
\end{verbatim}
\end{small}
Apart from the loop with the abstract machine instructions, the
emulator contains some pseudo-subroutines (entered by goto-
statements), e.g. for general unification\index{unification}, error handling etc. 
Moreover, a number of builtin predicates are coded inside the emulator. 
This is done because they need access to abstract machine registers
that would not be available outside the emulator, or just for improved
efficiency (e.g.\ basic arithmetic),

%Here is a rough map of the emulator file {\bf emu.c}:
%\begin{verbatim}
% 800 lines      declarations and macro definitions
% 100 lines      Initialization
% 350 lines      general unify and difference routines
% 500 lines      error and event handling code
%3500 lines      instructions
%1000 lines      built-ins
%\end{verbatim}
%

Closely related code is in the files:
\begin{description}
\item[code.c]
        Hand-coded sequences of abstract machine code, which implement
        certain built-in predicates, support event handling, etc.
\item[emu_util.c and emu_c_env.c]
        Supporting functions to execute low-level operations related
        to the abstract machine. Also debugging support.
\item[opcode.h] definition of the opcodes
\item[sepia.h] macros
\item[types.h] types
\item[emu_export.h] internal macros and definitions
\item[error.h] error codes
\end{description}


\subsection{Threaded Code}

\index{threaded code}
When {\eclipse} is built with the THREADED option, a threaded code
system is built. This is the usual configuration for releases.
However, it relies on a GCC feature for taking the address of
a code label (\&\&). The differences are summarised in the following
table. We consider the abstract machine instruction MoveAML:
\begin{small}
\begin{verbatim}
Non-threaded:                       Threaded:
-------------                       ---------

#define MoveAML 3                   #define MoveAML 3            
                                                                 
#define OpValue(x)  x               #define OpValue(x)  op_addr[x]

                                    op_addr[MoveAML] = &&I_MoveAML;    
                                                                 

Emulator code:                      Emulator code:

_loop_:
  switch (PP++->inst)
  {                                   {
        ...                                 ...

    case MoveAML:                       I_MoveAML:                   
        <code for instruction>              <code for instruction>       
        goto _loop_;                        goto *PP++->emu_addr;       

        ...                                 ...
  }                                   }
\end{verbatim}
\end{small}
OpValue() is a macro that is used by the code generator and
defines which actual op-code value is inserted into the generated code.
In the non-threaded case this is simply the instruction number,
in the threaded case, the value is the address of the emulator
code that implements the instruction.

Performance is significantly improved because the threaded transition
from one instruction to the next consists just in an indirect jump.


\subsection{Emulator Debugging}

The emulator\index{emulator}  code contains a few simple features that help debugging\index{debugging}
on the abstract machine level:
\begin{enumerate}
\item A circular buffer that records
the last MAX_BACKTRACE addresses of executed abstract instructions.
The backtrace can be printed by calling the C function lastpp(n).
\item A flag TRACE that will enable printing of abstract machine
instructions before they are being executed.
\item A variable stop_address that can be set (via c C debugger)
to an abstract code address where one wants to stop execution
(the emulator calls the function emu_break()).
\item A STATISTICS flag that enables counting of each
type of instruction.
\end{enumerate}
Theses facilities rely on an emulator loop being present and
are only available in the non-threaded version.

